\setcounter{tocdepth}{1}

\usepackage{emptypage}

\usepackage{fancyhdr} % To manage header and footer

\usepackage{fontspec} % Enable CJK
\usepackage{xeCJK} % Enable CJK

\setmainfont{Noto Sans CJK SC Light}
\setCJKmainfont{Noto Sans CJK SC Light}

\usepackage{tocloft} %Package for TOC
\usepackage[pdfpagelayout=TwoPageRight]{hyperref} % Force PDF to always open title page on right side
\usepackage{graphicx} % Only when need to insert diagram by eps file

\usepackage[% For page layout
	inner=25mm,
	outer=25mm,
%	textwidth=140mm,
	top=20mm,
	bottom=20mm,
	headheight=13pt,
	includehead,includefoot,
	heightrounded,
]{geometry}

\usepackage{xifthen}

\usepackage{titlesec}

\titleformat{\chapter}[display]
  {\Huge\bf}
  {\filleft 第 \thechapter 章}                % label
  {10pt}             % sep
  {\hrule\vspace{15pt}\filleft\huge}           % before-code

\pagestyle{fancy}

\newcommand{\partinheader}{\ifthenelse{\equal{\thepart}{I}}{古李十番棋}{\ifthenelse{\equal{\thepart}{II}}{古李对局记录}{ }}}

\fancyhead[RO]{\leftmark}
\fancyhead[LE]{\partinheader}
\fancyhead[LO,RE]{ }
%\fancyfoot{}

\renewcommand{\chaptermark}[1]{\markboth{\MakeUppercase{#1}}{}} % Remove "chapter." from header

\renewcommand\cftchapaftersnum{} % Change TOC chapter numbering for I. to I(nodot)
\renewcommand\cftsecaftersnum{} % Change TOC section numbering for 1. to 1(nodot)

\renewcommand\cftpartfont{\large\normalfont} % Change TOC Part font NOT to be bold
\renewcommand\cftpartpagefont{\normalfont} % Change TOC Part page number NOT to be bold

\renewcommand\cftchapfont{\normalfont} % Change TOC Chapter font NOT to be bold
\renewcommand\cftchappagefont{\normalfont} % Change TOC Chapter page number NOT to be bold

\renewcommand{\contentsname}{\hfill 目录\hfill}

\renewcommand\thesection{\arabic{chapter}} % Make section on TOC starts with 1,2,3

\renewcommand{\headrulewidth}{0.4pt}

\renewcommand{\baselinestretch}{1.3} % Set wider line spacing

\setlength\parindent{0pt} % Make all paragraph looks like no ident
\setlength\parskip{1em}

\newcommand{\dwitht}[5]{%
\begin{minipage}[t]{0.478\textwidth}
	\vspace{0pt} 
	\raggedright
	#1
	\centering
	\textbf{参考图
	#3 } -
	#4 
	的变化图 
	#5 
	\vspace{15pt} 
\end{minipage}
\hfill
\begin{minipage}[t]{0.5\textwidth}
	\vspace{4pt} 
	\raggedright
	#2
	\vspace{15pt} 
\end{minipage}
}

\newcommand{\maind}[3]{
	\begin{center}
	#1 \\
	\vspace{-2mm} \textbf{第 #2 谱} \ \ -( 第 #3  手 )\\
	\end{center}
}

\newcommand{\holder}{如果黑棋想直接占角的话,下在A位或者B位会是更为有效的选择。在\bsq 之后,白棋可直接在A位打入,并将角地占为己有。但是在白棋确定角地的过程中,黑棋会就向中腹建立起雄厚的外势,这又使得白棋难以掌握打入A位的时机。}

\input{stoneandmark}
