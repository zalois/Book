\chapter{一月}

\newpage


\section*{北京}

2014年1月26日,梦百合世纪之战古李十番棋的第一局在中国北京的深冬中揭开了帷幕。\\

北京作为中国各朝代的都城已有近千年的历史。长久以来的政治地位,令北京拥有了众多的古宫殿、古庙以及广阔的园林。\\

时至今日,北京已经以其庞大的服务业为主导,已经成为了中国乃至世界的经济重镇。除此之外,其深厚的文化底蕴也令其成为了无数文艺机构的摇篮。其中就包括了中国围棋协会。\\

在这一局棋之前,古李总共正面交锋了36次,各自的胜利次数为18比17,李世石稍占上风。而余下的一局则以千古难见的四劫连环平局收场(详见第37章)。在这十番棋开始之前,是很难断言谁将会是这次十番棋的最终胜利者的。\\

对许多棋迷来说,谁胜谁负并不是最重要的。能够见证两位围棋大家奕出一局局精彩的名局就已经让诸多棋迷感到心满意足了啊!\\

在这十番棋的日程里并没有正式的午餐时间,但是两位棋手可以随时离席享用赛方提供的餐点。\\

毫无疑问的,两位棋手都想在这第一局先驰得点,好将这十番棋的走向掌握在手中。猜先的结果是李世石这一局执黑,而接下来的棋局以互先的形式双方轮流执黑。
\newpage

\section*{星位}


李世石一开局就下在了星位。\\

在现代围棋中,\bs{1} 下在星位是最为常见的。这主要是因为下在星位的话,黑棋能够在目睹了白棋的开局之后更加灵活的选择应对的方式。\\

星位的优点是它的灵活性、速度、简约以及对中腹的影响力较大。而它最大的弱点就是它并没有很牢靠地将角地拿在手中,亦即角地的眼位还没能完全的确定下来。\\

如果黑棋想直接占角的话,下在A位或者B位会是更为有效的选择。在\bs{1} 之后,白棋可直接在A位打入,并将角地占为己有。但是在白棋确定角地的过程中,黑棋会就向中腹建立起雄厚的外势,这又使得白棋难以掌握打入A位的时机。\\

在这序盘阶段,白棋A位的打入还是嫌早了。不仅如此,白棋在接下来的一段时间里打入A位所能获得的利益也依旧是留有疑问的。有鉴于此,A位的打入其实暂时来说并不是那么严厉的,尤其是黑棋能在\bs{1} 周围发展出足以弥补角地损失的大模样是更是如此。\\

\newpage
\bs{1} \bs{2} \bs{3} \bs{4} \bs{5} \bs{6} \bs{7} \bs{8} \bs{9} \bs{10}
\ws{1} \ws{2} \ws{3} \ws{4} \ws{5} \ws{6} \ws{7} \ws{8} \ws{9} \ws{10}
%\bs{11} \bs{12} \bs{13} \bs{14} \bs{15} \bs{16} \bs{17} \bs{18} \bs{19} \bs{20}
%\ws{11} \ws{12} \ws{13} \ws{14} \ws{15} \ws{16} \ws{17} \ws{18} \ws{19} \ws{20}
%\bs{21} \bs{22} \bs{23} \bs{24} \bs{25} \bs{26} \bs{27} \bs{28} \bs{29} \bs{30}
%\ws{21} \ws{22} \ws{23} \ws{24} \ws{25} \ws{26} \ws{27} \ws{28} \ws{29} \ws{30}
%\bs{31} \bs{32} \bs{33} \bs{34} \bs{35} \bs{36} \bs{37} \bs{38} \bs{39} \bs{40}
%\ws{31} \ws{32} \ws{33} \ws{34} \ws{35} \ws{36} \ws{37} \ws{38} \ws{39} \ws{40}
%\bs{41} \bs{42} \bs{43} \bs{44} \bs{45} \bs{46} \bs{47} \bs{48} \bs{49} \bs{50}
%\ws{41} \ws{42} \ws{43} \ws{44} \ws{45} \ws{46} \ws{47} \ws{48} \ws{49} \ws{50}
